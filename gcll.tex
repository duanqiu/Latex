\documentclass[a4paper,12pt]{ctexart}
\title{工程师与工程伦理}
\author{戴强}
\date{\today}

\CTEXsetup[format={\Large\bfseries}]{section}


\renewenvironment*{abstract}[1]{%
\newcommand\gjc{#1}
\paragraph{摘要}
}{\paragraph{关键词:}\gjc }

% \newcommand\mytextcolor[1]{\textcolor{black}{#1}}

\usepackage[top=1.25in,bottom=1in,left=1.25in,right=1.25in]{geometry}
% \renewcommand{\chaptername}{第\CJKnumber{\thechapter}章}



\begin{document}

\maketitle
\newpage
\tableofcontents
\newpage
\quad
\quad

\begin{abstract}{工程,伦理,工程伦理,工程师}
摘要置于主体部分之前,目的是让读者首先了解一下论文的内容,以便决定是否阅读全文。一般来说,这种摘要在全文完成之后写。字数限制在100~150字之间。内容包括研究目的、研究方法、研究结果和主要结论。也就是说,摘要必须回答“研究什么”、“怎么研究”、“得到了什么结果”、“结果说明了什么”等问题。
简短精炼是学术期刊论文摘要的主要特点。只需简明扼要地将研究目的、方法、结果和结论分别用1~2句话加以概括即可。
\end{abstract}

\newpage
\section{引言}
随着现代科学技术的发展,现代工程在人类生活中所占的比重越来越大,而越来越复杂的现代技术则需要更多地依托工程来实现。随着工程技术的不断发展,工程技术的负面效应也日渐突出 。环境污染 、能源危机等一系列问题的出现, 使得与工程技术联系最为密切的工程伦理问题成为工程界 、哲学界和社会广泛关注的问题。
现代工程活动使工程师扮演了一个极其重要的专业角色 ,工程自身的技术复杂性和社会联系性 ,必然要求工程师不仅精通技术业务 ,能够创造性地解决有关技术难题 , 还要善于管理和协调 ,处理好与工程活动相联的各种关系 。

作为工程师预备队的我们当代学生,就有必要探讨一下工程师与工程伦理的关系,以及工程师作为工程活动的主体,在工作过程中会遇到的各种伦理问题以及如何解决等工程伦理问题,本文将就此展开并就具体例子进行说明分析。
\section{工程师与工程伦理的论述}
工程是人类有明确目的的造物过程及其结果。造物的主体是人,造物过程是工程活动, 造
物的结果是成形的具有特定功能的物\footnote{对工程伦理的几点思考,一、基本概念}

伦理一般来说是阐述、分析人与人之间关系的道理。如中国古代的君君、臣臣、父父、子子,近代的平等、公平、尊重等。近代以来伦理也进一步推广为人与外界,包括人与他人、人与物,以至人与环境之间的关系。

工程伦理是阐述、分析工程(包括活动和结果)与外界之间的关系的道理。是从工程问题中推演出来的,把工程问题提到道德高度, 既有助于提高工程技术人员的道德素质和道德水平 ; 又有助于保证工程质量 ,最大限度地避免工程风险。

工程师作为工程活动的主体 , 在工作过程中会遇到各种伦理问题 。这就使得工程师在对工程进行设计,实施时就必须要承担相应的社会伦理责任。伦理责任的含义是指人们要对自己的行为负责 , 该行为是可以答复和解释说明的 。相对于法律责任而言 , 伦理责任具有前瞻性 , 它是一种以善与恶 、正义与非正义 、公正与偏私 、诚实与虚伪 、荣誉与耻辱等作为评判准则的社会责任。

在中国古代很少提及工程师的伦理责任,而现在却越来越多的被提及被强调,最主要的原因可以用《Spider-Man》里面的一句台词来概括:With great power comes great responsibility(能力越大责任越大)。工程师的这个职业有可能事关人类的前途和命运的选择 。在传统观点中 ,工程师的社会责任是做好本职工作,这是因为原来的工程很少有现代刚才这么多的不确定性,关乎如此多的生活,造成巨大并深远的影响,因此不需要过分强调工程师的伦理责任,而这种看法放到现在的实际情景中就显得有点片面了 。如引言中所述 , 当代工程技术的新发展赋予科技工作者前所未有的力量 ,使他们的行为后果常常大到难以预测 ,信息技术 、基因工程等工程技术在给人类带来利益的同时还带来可以预见和难以预见的危害甚至灾难 , 或者给一些人带来利益而给另一些人带来危害 。如美国电影《生化危机》中应为病毒泄露而出现大量的丧尸,《猩球崛起》中造成人类灭绝的猿流感等,虽然都是想象,却越可以从中看到现代工程师的责任重大。可见 ,在现代社会 ,工程师在某些时候伦理责任可能还要远远超过做好本职工作。

就目前工程师在工程中应该以人为本,关爱生命。确保工程能有效的为人类服务而避免可能的伤害。关爱自然,可持续发展。不从事和开发可能破坏生态环境或对生态环境有害的工程 ,工程师进行的工程活动要有利于自然界的生命和生态系统的健全发展 ,提高环境质量 。要在开发中保护 ,在保护中开发 。而不能为了经济效益而肆意破坏环境,或者为了短期利益造成不可挽回的影响,为了当代人的利益而损害下一代人的利益。

\section{案例与分析}
工程职业与工程职业伦理在真实世界中的关系往往不是那么清晰明了。工程实践中具体情况的复杂性往往会超出理论预设范围,在时间过程中常有很多模棱两可或例外的情况出现。因此工程师的伦理责任往往不是口头上说要爱岗,敬业,对社会负责,对自然负责等几个概念性的词汇就可以的。工程师也常常面临两难的选择。而选择和责任是分不开的 , 选择将工程师带进价值冲突之中 ,使他们在多种可能性中取舍。因此我将结合几个典型案例及其分析来表述在实际情景中我对工程师与工程伦理的理解。
\subsection{案例1:斯诺登与菱镜门}
2013年6月,前中情局(CIA)职员爱德华·斯诺登将两份绝密资料交给英国《卫报》和美国《华盛顿邮报》,并告之媒体何时发表。按照设定的计划,6月5日,英国《卫报》先扔出了第一颗舆论炸弹:美国国家安全局有一项代号为"棱镜"的秘密项目,要求电信巨头威瑞森公司必须每天上交数百万用户的通话记录。6月6日,美国《华盛顿邮报》披露称,过去6年间,美国国家安全局和联邦调查局通过进入微软、谷歌、苹果、雅虎等九大网络巨头的服务器,监控美国公民的电子邮件、聊天记录、视频及照片等秘密资料。美国舆论随之哗然。

而为了躲避美国政府的通缉,斯诺登从夏威夷逃亡到香港,并于同年6月从香港前往莫斯科。在莫斯科机场长时间逗留后,他终于获得了在俄罗斯避难一年的许可。

在这一年中,斯诺登的“身份”一直捉摸不定:多数网民认为他是“英雄”,向大众公开了真相,美国政府则称他“背叛祖国”,并非告密者和人权活动人士,而是受到一系列“严重刑事指控”的通缉犯。

在这一年中,斯诺登数次现身媒体,不断地曝光美国政府的监听计划,让并于2013年圣诞节发表讲话:“团结起来,我们可以找到更好的平衡,终结政府监控。”
\subsubsection{案例1分析}
斯诺登的行为一直争议不断,有人称他为英雄,因为他为了大众的隐私而敢于站出来揭露美国政府的恶行。但有人认为他是叛国者并且没有职业素养,参与到被政府要求保密的计划中却将计划披露出来,损害国家的利益。站在斯诺登的角度看这是一个两难困境。一方面不忍公民的隐私被侵害,另一方面又理解国家为了找出恐怖分子保护公民的用意,并且被要求保密的职业素养。

就笔者本人看来,斯诺登做出来正义的并且他人想做而不敢做的行为,他是一个英雄。作为一个现代工程师,不仅仅要对职业本身负责,还要承担起对社会的责任。美国政府在攻击中国的时候口口声声人权至上,民主,自由。而美国政府的合法性是建立在对宪法的遵守上的。而美国政府秘密监听民众的通信记录,违法了大众认可的价值,在这件事不具有合法性。美国政府对一种社会价值进行侵犯。在美国这么一个个人隐私侵犯都要上法庭的国家,对整个社会民众隐私进行侵犯,可见起情节严重。这绝不是美国政府所谓的背叛祖国。相反,这是为了维护国家的价值,找到更好的平衡。如果美国真的为了查出犯罪分子,也应该进行全名投票。

   当对方的的的


\subsection{案例2:黄万里与三峡工程}
三峡大坝自1994年动工以来,一直饱受争议,其中发对声音最响的要数中国水利专家黄万里老先生了。老先生多次谏言修建三峡大坝的危害以及影响民生。其实在此之前,黄老先生就应为反对三门峡工程而闻名。

1954年,中国从苏联请来专家,帮助规划黄河治理。这些苏联专家是搞水利工程的,擅长修筑坚固的水坝。他们经过研究后就决定选址三门峡建大坝,他们认为三门峡是个难得的好坝址。1955年4月,黄河三门峡大坝工程动工。同月,中国水利部召集学者和水利工程师70多人就已开工的黄河三门峡水利规划方案进行讨论。黄万里是两个反对苏联专家的与会者之一,并与其他专家在会上进行了七天的辩论。最终现实表明了黄万里先生是正确的。由于在前苏联没有黄河这么大的水土流失量,所以苏联专家在设计时忽视了这个问题。最终水库库尾泥沙淤积,造成渭河入黄河部分抬高(甚至泥沙倒灌),渭河下游洪患严重、土地盐渍化,不得不降低蓄水位运行,并按蓄清排浑运用。同时水库由于控制淹没损失一再缩小设计蓄水位,又在泥沙严重淤积后严重损失库容,不能满足黄河洪水控制的规划要求,水利枢纽由于降低蓄水位运用和反复改建浪费大量投资,发电效益因水头降低和泥沙磨蚀大减。在后来的改建中,有不得不将黄万里力主不要堵死、却依然被堵死的原坝底6个排水孔,以每个一千万元的代价重新炸开。黄万里先生在有生之年,看到自己对三门峡的意见不幸言中,于是坚决反对三峡大坝的建设。2001年8月27日下午3时5分,在清华大学校医院一间简朴的病房,90岁的黄万里先生溘然而逝。生前卧病,90岁高龄的他几番涕泪纵横,反反复复只有一句话:“他们没有听我一句话啊!”

\subsubsection{案例1分析}
三峡大坝建成至今众说纷纭。有人说弊大于利,黄万里有远见。也有人说是利大于弊,黄万里是瞎担心,他所说的危害都没怎么出现。并且貌似后一种看法的人居多。但我认为黄老先生是值得敬佩的。他的意义在于当国家都在说要上的时候,他能根据自己的专业知识,冒着被划为右派的风险(最终因为这件事被划为右派),提出自己的反对意见。这里面包含这黄老先生对工程伦理道德的坚守,承担起对社会,对自然,对国家的责任。

黄万里对于工程可能带来的灾害的担忧是有道理的,有责任心的人应该担忧这些事情。因为工程的实施都是利弊相伴的,需要有人通过各方面的综合思考提出反面意见来促使人们进行更多地思考。而国家对水坝工程中的反面意见的重视程度是相当的高,最终建造的时候也是谨慎再谨慎。

孔子曰:见义不为,无勇也。黄万里先生是一个勇士,在其自身领域而去努力惠及人民,影响当政者的决策至利民的地步。而这中精神也是值得我们学习的。当我们以后遇到伦理困境的是时候,也可以拿来激励自己,坚守住自己的职业底线,不仅要对工程本身负责,也要对社会,对自然以及后续影响负责。



\section{结语}

通过对上述案例的讲述,就是为了强调一个意识,那就是工程师不仅仅要关注职业本身,还要肩负起对社会,对国家的责任。在现代社会中,大家对工程师的伦理责任意识还很淡薄,如快播事件,明显加快了不健康内容的传播,影响了青少年的身心健康,但大部分职员只认为这是一个工作,没必要承担如此大的责任,也没有哪个职员提出异议。因此我们必须明确工程师在工程伦理中承担的责任,加强自己职业道德建设,让工程能够更好的为人类服务,避免工程给人类带来的伤害。

\end{document}
