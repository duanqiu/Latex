\documentclass{ctexart}
\title{基于PCA与SVM的人脸识别算法分析}
\author{戴强}
\date{\today}

\renewenvironment*{abstract}[1]{%
\newcommand\gjc{#1}
\paragraph{摘要}
}{\paragraph{关键词:}\gjc }


\begin{document}

\maketitle

\section{摘要}
\label{sec:zhaiyao}

\begin{abstract}{PCA,SVM,人脸识别,生物模式识别}
摘要置于主体部分之前,目的是让读者首先了解一下论文的内容,以便决定是否阅读全文。一般来说,这种摘要在全文完成之后写。字数限制在100~150字之间。内容包括研究目的、研究方法、研究结果和主要结论。也就是说,摘要必须回答“研究什么”、“怎么研究”、“得到了什么结果”、“结果说明了什么”等问题。
简短精炼是学术期刊论文摘要的主要特点。只需简明扼要地将研究目的、方法、结果和结论分别用1~2句话加以概括即可。
\end{abstract}


\newpage

\tableofcontents


\section{前言}
\label{sec:qianyan}

\section{算法分析}

\section{程序实现}
\label{sec:shixian}

\section{结果分析}
\label{sec:jieguofenxi}




\end{document}







