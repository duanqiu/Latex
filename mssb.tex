\documentclass{ctexart}
\title{基于PCA与SVM的人脸识别算法分析}
\author{戴强}
\date{\today}
\usepackage{amsmath}
\usepackage{listings}
\lstset{language=Matlab}

\renewenvironment*{abstract}[1]{%
\newcommand\gjc{#1}
\paragraph{摘要}
}{\paragraph{关键词:}\gjc }


\begin{document}

\maketitle

\section{摘要}
\label{sec:zhaiyao}

\begin{abstract}{PCA,SVM,人脸识别,生物模式识别}
摘要置于主体部分之前,目的是让读者首先了解一下论文的内容,以便决定是否阅读全文。一般来说,这种摘要在全文完成之后写。字数限制在100~150字之间。内容包括研究目的、研究方法、研究结果和主要结论。也就是说,摘要必须回答“研究什么”、“怎么研究”、“得到了什么结果”、“结果说明了什么”等问题。
简短精炼是学术期刊论文摘要的主要特点。只需简明扼要地将研究目的、方法、结果和结论分别用1~2句话加以概括即可。
\end{abstract}


\newpage

\tableofcontents

\newpage


\section{概论}

\subsection{相关的}

\subsection{使用的人脸库}

face94人脸图像样本库是Essex大学的彩色人脸图像库,共3060张图像,153个人,每人20张,每张大小均为200x180,图像库中的人脸图像在人脸位置、光照和表情方面有一定的变化,其中一些人戴着眼镜或留有胡须。
500个人脸样本 识别率为99\%。

ORL人脸库是由Ohvcrtti实验室拍摄的一系列人脸图像组成,它是一个美国的专门从事人脸图像研究的实验室,该数据库内共有40个不同年龄、不同性别和.不同种族的对象,每个对象1 O幅图像。这里面的人脸图像不是彩色的,二是灰度图像,尺寸为11 2×92,背景为灰色。它的人脸部分表情和细节都是不同的而且是要求变化的,比如哭与不哭,有没有头发,有没有戴帽子等等,同时,人脸姿态也有变化,人脸的尺寸也有最多1 0%的变化。
200个人脸样本 识别率为83.5\%。


Yale耶鲁人脸库,该数据库包含15人每人11幅人脸像,其图像差异在于变化的人脸表情,是否戴有眼镜,不同的光照条件等。该库的特点就是光照变化显著,且有面部部分遮掩。在Yale人脸库中包含15个人的165幅图像

75个人脸样本 识别率85.333\%.


本项目分别使用两种人脸库来进行比较分析。

\subsection{本文大致安排}



\section{算法分析}

\subsection{PCA降维}

\subsubsection{训练样本归一化}

将图片的二维像素展开成一维行向量,所有测试图片累加,组成$M*N$的向量,其中M表示图片的个数,N表示一副图片的像素点个数
\[
\begin{bmatrix}
A_{11} & A_{12} & \ldots & A_{1n}\\
Ax_{21} & A_{22} & \ldots & Ax_{2n}\\
\vdots & \vdots & \ddots & \vdots\\
A_{m1} & A_{m2} & \ldots & A_{mn}\\
\end{bmatrix}
\]

然后得到每个相应位置的像素点均值与方差,进行归一化,matlab的代码实现如下:

\begin{lstlisting}
A_mean = mean(A);
A_norm = (A-repmat(A_mean, m, 1));
A_sigma= std(A_norm)
A_norm = A_norm./A_sigma
\end{lstlisting}

\subsubsection{提取特征值}

根据PCA的算法要求,我们要求$A^T*A$的最大几个特征值的特征特征向量。用face94人脸库时,此时的矩阵$A$规模为153*36000。则我们要求的关联矩阵$A^T*A$的规模为36000*36000,这是一个非常大的计算量。因此先求$A*A^T$的最大几个特征值的特征特征向量(此时的矩阵规模为153*153),然后用求得的特征向量去乘以$A^T$。下面是相关证明:

1)$A^T*A$与$A*A^T$具有相同的秩:

若$A*x=0$,则$A^TA*x=0, x^TA^\mathrm{T}A*x=0 $

\quad
2)$A^T*A$与$A*A^T$具有相同的秩:

\quad
3)若$A*A^T$有特征向量$x$,且相应的特征值为$\lambda$,那么$A^Tx$为$A^T*A$的特征向量,并且相应的特征值同为$\lambda$:

\section{程序实现}

\section{结果分析}







\end{document}







