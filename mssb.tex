\documentclass[a4paper,12pt]{ctexart}
\title{基于PCA与SVM的人脸识别算法分析}
\author{戴强}
\date{\today}
\usepackage{amsmath}
\usepackage{listings}
\lstset{language=Matlab}
\usepackage{graphicx}
\usepackage{caption,subcaption}
\usepackage[top=1.25in,bottom=1in,left=1.25in,right=1.25in]{geometry}
\renewenvironment*{abstract}[1]{%
\newcommand\gjc{#1}
\paragraph{摘要}
}{\paragraph{关键词:}\gjc }


\begin{document}

\maketitle

\newpage
\section{摘要}
\label{sec:zhaiyao}

\begin{abstract}{PCA,SVM,人脸识别,生物模式识别}
  本文主要详细论述了基于PCA与SVM的实现过程,并将程序运用于不同人脸库进行对比测试。研究不同图片对该算法的影响,及该算法性能算法分析。
\end{abstract}


\newpage


\tableofcontents

\newpage


\section{概论}
由于PCA在特征提取方面的有效性以及SVM在处理小样本问题泛化能力强等方面的优势,PCA提取数据然后使用SVM进行分类是一种非常经典的人脸识别算法。本设计算法是对其的实现并用不同的人脸库进行人脸识别,来评估该算法的性能,并通过不同的人脸库来探究不同的图片数据库对人脸识别的新能的影响。

\subsection{程序流程图}
该算法分为两部分:

1) 用数据来对程序进行训练,得到相应参数,其流程如图\ref{fig:lct:train}

2) 用训练好的参数进行人脸识别,相应流程图如图\ref{fig:lct:test}
\begin{figure}[htb]
\begin{minipage}[t]{0.5\linewidth}
\centering
\includegraphics[width=1in]{../Picture/PM_train.jpg}
\caption{数据训练流程图}
\label{fig:lct:train}
\end{minipage}%
\begin{minipage}[t]{0.5\linewidth}
\centering
\includegraphics[width=1in]{../Picture/PM_test.jpg}
\caption{数据分类流程图}
\label{fig:lct:test}
\end{minipage}
\end{figure}

\subsection{人脸库}

本实验使用了3种人脸库来进行人脸识别,它们分别为:

1)\quad face94人脸图像样本库是Essex大学的彩色人脸图像库,共3060张图像,153个人,每人20张,每张大小均为200x180,图像库中的人脸图像在人脸位置、光照和表情方面有一定的变化,其中一些人戴着眼镜或留有胡须。
%500个人脸样本 识别率为99\%。

2)\quad ORL人脸库是由Ohvcrtti实验室拍摄的一系列人脸图像组成,它是一个美国的专门从事人脸图像研究的实验室,该数据库内共有40个不同年龄、不同性别和.不同种族的对象,每个对象1O幅图像。这里面的人脸图像是灰度图像,尺寸为112×92,背景为灰色。它的人脸部分表情和细节都是不同的而且是要求变化的,比如哭与不哭,有没有头发,有没有戴帽子等等,同时,人脸姿态也有变化,人脸的尺寸也有最多10%的变化。
%200个人脸样本 识别率为83.5\%。


3)\quad Yale人脸库,该数据库包含15人每人11幅人脸像,其图像差异在于变化的人脸表情,是否戴有眼镜,不同的光照条件等。该库的特点就是光照变化显著,且有面部部分遮掩。在Yale人脸库中包含15个人的165幅图像。

%75个人脸样本 识别率85.333\%.

\subsection{论文结构}
本论文共分为4章

第一章为概要,对算法进行大体的流程图进行展示,并介绍所使用的人脸库,并对论文的总体结构进行讲述。

第二章为算法分析与实现,先对PCA算法的部分细节进行讲述,以及论述SVM在多分类情形的结构构建。

第三章为结果分析,对结果进行展示。并比较该程序在不同的人脸库下的表现并进行分析。

第四章为总结,对所做的工作进行总结,并阐述了下一步的研究工作。

\newpage
\section{算法分析与实现}

\subsection{PCA降维}
PCA是一种经典的特征提取及数据表示方法,被广泛用于模式识别领域和计算机视觉领域。

由于人脸图像样本的维数很高,若直接对人脸图像进行处理则计算量很大并且运算时间很长,所以近年来,很多人脸检测人脸识别算法都首先采用PCA方法进行降维,然后再采用其他方法进行进一步处理。

\subsubsection{训练样本归一化}

将图片的二维像素展开成一维行向量,所有测试图片累加,组成$M*N$的向量,其中M表示图片的个数,N表示一副图片的像素点个数
\[
\begin{bmatrix}
A_{11} & A_{12} & \ldots & A_{1n}\\
Ax_{21} & A_{22} & \ldots & Ax_{2n}\\
\vdots & \vdots & \ddots & \vdots\\
A_{m1} & A_{m2} & \ldots & A_{mn}\\
\end{bmatrix}
\]

然后得到每个相应位置的像素点均值与方差,进行归一化,matlab的代码实现如下:

\begin{lstlisting}
A_mean = mean(A);
A_norm = (A-repmat(A_mean, m, 1));
A_sigma= std(A_norm)
A_norm = A_norm.repmat(A_sigma, m, 1)
\end{lstlisting}

\subsubsection{提取特征值}

根据PCA的算法要求,我们要求$A^\mathrm{T}*A$的最大几个特征值的特征向量。用face94人脸库时,此时的矩阵$A$规模为1530*36000。则我们要求的关联矩阵$A^\mathrm{T}*A$的规模为36000*36000,这是一个非常大的计算量。因此先求$A*A^\mathrm{T}$的最大几个特征值的特征特征向量(此时的矩阵规模为1530*1530),然后用求得的特征向量去乘以$A^\mathrm{T}$,同样可得到$A^\mathrm{T}*A$的最大几个特征值的特征向量。下面是相关证明:

1)$A^\mathrm{T}A$与$AA^\mathrm{T}$具有相同的秩:

若$Ax=0$\quad $\Longrightarrow$ \quad$A^\mathrm{T}Ax=0 $

若$A^\mathrm{T}Ax=0 $ ,则 $x^\mathrm{T}A^\mathrm{T}Ax=0 $,$(Ax)^\mathrm{T}Ax=0$ \quad $\Longrightarrow$ \quad $A*x=0$

综上所述,$Ax=0$与$A^\mathrm{T}Ax=0$通解 \quad $\Longrightarrow$ \quad $r(A)=r(A^\mathrm{T}A)$

同理可得$r(A)=r(AA^\mathrm{T})$。

因此$r(AA^\mathrm{T})=r(A^\mathrm{T}A)$
\newline

2)$A^\mathrm{T}A$与$AA^\mathrm{T}$具有相同的特征值:

$\left| AA^\mathrm{T}-\lambda E  \right|=0$ \quad $\Longrightarrow$   $\left| (A^\mathrm{T}A-\lambda E)^\mathrm{T}  \right|=0$
\quad $\Longrightarrow$  $\left| A^\mathrm{T}A-\lambda E  \right|= 0$

因此,$A^\mathrm{T}A$与$AA^\mathrm{T}$具有相同的特征值。
 \quad
 \newline
 
 3)若$AA^\mathrm{T}$有特征向量$x$,且相应的特征值为$\lambda$,那么$A^\mathrm{T}x$为$A^\mathrm{T}A$的特征向量,并且相应的特征值同为$\lambda$:

 已知$AA^\mathrm{T}x=\lambda x$,那么$A^\mathrm{T}A*A^\mathrm{T}x=A^\mathrm{T}*AA^\mathrm{T}x=A^\mathrm{T}*\lambda x$

 即$A^\mathrm{T}A*A^\mathrm{T}x=\lambda A^\mathrm{T} x$

 因此$A^\mathrm{T}x$为$A^\mathrm{T}A$的特征向量,并且相应的特征值同为$\lambda$。
 
\begin{lstlisting}
相应的matlab代码为:
covMatT = A_norm * A_norm';
% 计算 covMatT 的前 k 个特征值和特征向量
[V D] = eigs(covMatT, k);
% 得到协方差矩阵 (covMatT)' 相应的特征向量
V = A_norm' * V;
% 特征向量归一化为单位特征向量
for i=1:k
    V(:,i)=V(:,i)/norm(V(:,i));
end
% 线性变换(投影)将原矩阵降维至 k 维
pcaA = A_norm * V;
 \end{lstlisting}
 本程序取前20个特征向量,那么原样本数据的规模为153*36000,降维后的矩阵为153*20。即每幅图片由原来的36000个数据点降维至20个数据点,大大减少了数据量,使得SVM分类减少了难度。

 % 20个特征向量 yale 15个人 pca贡献率0.8944  准确率 85.333%
 % 20个特征向量 ORL 15 个人 pca贡献率0.8283  准确率92%  40个人  贡献率0.73 准确率83.5
 % 30个特征向量  ORL 40个人  0.7891  84.5%
 % 40                40      0.8289   85.5
 % 50                40      0.8587
 
 \subsection{SVM分类}
 支持向量机是Vapnik提出的旨在改善传统神经网络学习方法的理论弱点(如结构的确定靠试验试凑,没有理论依据),最先从最优分类面问题提出了支持向量机网络,目前它可用在模式识别、数据挖掘等方面。

 由于经典的SVM算法只支持二分类,而人脸识别是多分类问题。用它来实现人脸识别就必须增加它的结构以实现多分类。基本策略有2种:

 1) “一对多策略”:为每个类别构建一个分类器,将该类的数据确定为正样本,将其它类的数据统一划分为负样本加以训练。
 
 2) “一对一策略”:为每两个类别中间构建一个分类器,那么解决K分类问题就需要$K(K-1)/2$个分类器。

 
 本程序使用的是“一对多策略”,任意两个不同人之间用一个SVM分类器来进行分类,当识别人脸的时候就将人脸数据放到每个分类器中进行分类,每个分类器一票进行投票,投票数最多的那个人就被认为是最终所识别的那一个。如图\ref{fig:msvm}所示就是一个三分类SVM结构。
 

\begin{figure}[!htb]
  \centering
  \includegraphics{../Picture/tree_class.jpg}
  \caption{多分类SVM结构图}
  \label{fig:msvm}
\end{figure}

\newpage
\section{结果及其分析}
下面的主要内容为展示与分析实验结果,分为人脸库间的比较与图片库内部的比较来分析。结果中的主要数据项有:人脸库,训练与测试的人数(训练的人数与测试的人数一致,取每个人的前5张图片训练,取每个人的后5张图片用于测试),特征个数为通过PCA降为后剩余的特征数。PCA贡献率的算法为PCA贡献率=选取的特征向量对应特征值的和/所有特征向量对应特征值的和。准确率为测试总样本的准确率。


\subsection{各人脸库结果分析}

\subsubsection{ORL人脸库结果及分析}

ORL人脸库的图片尺寸为112*92,样本如图\ref{pic:orl}所示,同一个人脸图片的差别主要是人脸的角度具有差异性:仰头,低头,左侧脸,有侧脸的差别。

\begin{figure}[ht]
  \centering
  \centerline{\includegraphics{../Picture/composite_ORL.jpg}}
  \caption{ORl人脸库样本}
\label{pic:orl}
\end{figure}


测试结果如表\ref{tab:orl}所示。


\begin{table}[!htb]
  \centering
\begin{tabular}{|r|c|c|c|c|}
\hline
人脸库&人数&特征个数&PCA贡献率&准确率\\
\hline
ORL&    40&     30&     0.7891& 84.5\\
ORL&    40&     40&     0.8289& 85.5\\
ORL&    40&     50&     0.8587& 88\\
ORL&    40&     60&     0.8814& 87.5\\
  ORL&    40&     70&     0.9000& 88\\
  \hline
ORL&    15&     20&     0.8283  &85.5\\
ORL&    15&     30&     0.8903  &96\\
ORL&    15&     40&     0.9296  &97.333\\
ORL&    15&     50&     0.9584  &97.333\\
  \hline
\end{tabular}
  \caption{ORL测试结果}
  \label{tab:orl}
\end{table}

可以看到一个图片的初始特征数为112*92,当训练与测试的人脸为40人时,经过PCA降维后30个特征就可以包含其中78.91\%的内容,70个特征就可以包含其中90\%的内容。开始随着特征数量的增加准确率逐步提高,当提取的特征数为50时,准确率为88\%,此后,结果收敛,增加特征值准确率不再升高。该算法识别率最高为88\%。当人脸数的人数减少到15人时,30个特征向量就包含89\%的内容,50个特征就可以包含其中95.84\%的内容,而准确率也高达97\%。。可以看出人数越少,特征值的贡献率越高,准确率越高。同一张人脸不同角度的变化对该算法识别率有较大影响。当人数变多时不同角度的人脸图片容易被误判。该算法在该人脸库上,人数越多,识别率越下降。

\subsubsection{Yale人脸库结果及分析}

yale的人脸库的图片尺寸为168*192,样本如图\ref{pic:yale}所示,同一个人的不同头像的差异主要表现在:有没有戴眼镜,表情,是否闭眼,与拍摄光线的差别。

\begin{figure}[!htb]
  \centering
  \centerline{\includegraphics[width=10cm]{../Picture/composite_yale.jpg}}
  \caption{YALE人脸库样本}
\label{pic:yale}
\end{figure}

测试结果如表\ref{tab:yale}所示。
\begin{table}[!htb]
  \centering
\begin{tabular}{|r|c|c|c|c|}
\hline
人脸库&人数&特征个数&PCA贡献率&准确率\\
\hline
yale&   15&     20&     0.8944& 85.333\\
yale&   15&     30&     0.9428& 88\\
yale&   15&     40&     0.9677& 90.6667\\
yale&   15&     50&     0.9830& 90.6667\\
  \hline
\end{tabular}
  \caption{yale测试结果}
  \label{tab:yale}
\end{table}

由于yale数据库只有15个人的人脸图像,因此只测试了15个人的数据,从结果中可以看出PCA对该算法有比较好的压缩率,20个特征就可以包含其中89\%的内容,50个特征就可以包含其中98.3\%的内容。但准确率不那么令人满意。表情,眼睛,光线对人脸识别的影响很大。


\subsubsection{face94人脸库结果及分析}

face94的人脸库图片尺寸为200*180,face94人脸库样本如下图所示。同一个人的不同样本的不同差别非常小,人脸角度基本就没什么变化,只有面部表情的微小变化,如张嘴闭嘴,睁眼闭眼,左看右看等。

\begin{figure}[htb]
\begin{minipage}[t]{0.5\linewidth}
\centering
\includegraphics[width=2.5in]{../Picture/composite_face95.jpg}
\caption{face94第一个人脸样本}
\end{minipage}%
\begin{minipage}[t]{0.5\linewidth}
\centering
\includegraphics[width=2.5in]{../Picture/composite_face95_2.jpg}
\caption{face94第二个人脸样本}
\end{minipage}
\end{figure}



测试结果如表\ref{tab:face94}所示。
\begin{table}[!htb]
  \centering
\begin{tabular}{|r|c|c|c|c|}
\hline
人脸库&人数&特征个数&PCA贡献率&准确率\\
\hline
face94 & 150&    20&     0.7476& 99.333\\
  face94&  150&    10&     0.6483& 99.6\\
  \hline
face94&  100&    20&     0.7657& 99\\
face94 & 100&    10&     0.6583& 99.8\\
face94  &100&    5&      0.5430& 97\\
face94 & 100&    3&      0.4586& 91.2\\
  \hline
\end{tabular}
  \caption{face94测试结果}
  \label{tab:face94}
\end{table}

从结果中可以看出pca对该算法有非常好的识别率,在有100人的人脸数时,3个特征向量就包含45\%的内容,而识别率却在91.2\%,10个特征向量包含65\%的内容,但准确率高达99.8\%。这个结果证明该算法对该人脸库的效果特别好。有如此高的准确率的重要原因是该人脸库类内的差距非常大,而类间的差距非常小。是需要极少的特征就可以将不同的人分清楚。


\subsection{各人脸库综合分析}

各人脸库的综合结果如表\ref{tab:zhonghe}所示:

\begin{table}[!htb]
  \centering
\begin{tabular}{|r|c|c|c|c|}
\hline
人脸库&人数&特征个数&PCA贡献率&准确率\\
\hline
yale&   15&     20&     0.8944& 85.333\\
yale&   15&     30&     0.9428& 88\\
yale&   15&     40&     0.9677& 90.6667\\
yale&   15&     50&     0.9830& 90.6667\\
\hline
ORL&    15&     20&     0.8283  &85.5\\
ORL&    15&     30&     0.8903  &96\\
ORL&    15&     40&     0.9296  &97.333\\
ORL&    15&     50&     0.9584  &97.333\\
\hline
face94& 15&     10&     0.8315& 94.6667\\
face94& 15&     20&     0.9474& 96\\
face94& 15&     30&     0.9757& 98.667\\
\hline
\end{tabular}
  \caption{各人脸库综合分析结果表}
  \label{tab:zhonghe}
\end{table}

当把各个人脸库的人数都限定为15人的时候,我们就可以纵向进行比较各因素对人脸库的影响。综合看来该算法在face94上的表现最好,这也是理所当然的,因为各个人的图片差别很小(在训练集与测试集上),而不同人的图片差别非常大。ORL的表现次一点,说明不同角度的图像对该算法有一定的影响,但影响不够大。表现最差的是yale的人脸库,说明有无眼睛,光线等因素对该算法有较大的影响。

结合上面各个库的结果表,当人数增多的时候face94的表现没有上面差别都能很好的进行区分,ORL库人数增多时有较大的影响,当15人的时候ORl的最好识别率为97.333\%。当增加到40人时,最好的识别率只有88\%。人数越多,识别率越低。说明该算法对较小的数据时,有良好的表现,对较多的数据时,识别能力不足。

\section{总结}

本论文对PCA降维然后用SVM进行分类的人脸识别算法进行详细的讲述。通过对结果的分析,我们可以看到它在小样本数据时表现良好,并且运算速度快,对大样本或者各个不同角度的图片分类就不够准确。对比现在火热的神经网络方法的人脸识别,其对大样本的数据表现优秀,对小样本的数据表现不好,而且运算速度慢。

现在是大数据时代,数据量不是问题,而硬件的发展也使得神经网络计算的越来越快。因此,现在的主流的人脸识别算法是基于神经网络的人脸识别,而本文中所述的算法基本已经淘汰。但通过该算法的实现,手动的完成了一个基本人脸识别算法,了解了一些基本的生物特征识别的知识,学习了人脸识别算法的思想。有很多有益的收获。

愿生物特征识别能够更好的造福人类,让人人都享受科技的便利。

\end{document}



