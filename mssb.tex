\documentclass{ctexart}
\title{基于PCA与SVM的人脸识别算法分析}
\author{戴强}
\date{\today}

\renewenvironment*{abstract}[1]{%
\newcommand\gjc{#1}
\paragraph{摘要}
}{\paragraph{关键词:}\gjc }


\begin{document}

\maketitle

\section{摘要}
\label{sec:zhaiyao}

\begin{abstract}{PCA,SVM,人脸识别,生物模式识别}
摘要置于主体部分之前,目的是让读者首先了解一下论文的内容,以便决定是否阅读全文。一般来说,这种摘要在全文完成之后写。字数限制在100~150字之间。内容包括研究目的、研究方法、研究结果和主要结论。也就是说,摘要必须回答“研究什么”、“怎么研究”、“得到了什么结果”、“结果说明了什么”等问题。
简短精炼是学术期刊论文摘要的主要特点。只需简明扼要地将研究目的、方法、结果和结论分别用1~2句话加以概括即可。
\end{abstract}


\newpage

\tableofcontents


\section{前言}
\label{sec:qianyan}


\section{概论}



\subsection{人脸库}

face94人脸图像样本库是Essex大学的彩色人脸图像库,共3060张图像,153个人,每人20张,每张大小均为200x180,图像库中的人脸图像在人脸位置、光照和表情方面有一定的变化,其中一些人戴着眼镜或留有胡须。

ORL人脸库是由Ohvcrtti实验室拍摄的一系列人脸图像组成,它是一个美国的专门从事人脸图像研究的实验室,该数据库内共有40个不同年龄、不同性别和.不同种族的对象,每个对象1 O幅图像。这里面的人脸图像不是彩色的,二是灰度图像,尺寸为11 2×92,背景为灰色。它的人脸部分表情和细节都是不同的而且是要求变化的,比如哭与不哭,有没有头发,有没有戴帽子等等,同时,人脸姿态也有变化,人脸的尺寸也有最多1 0%的变化。

本项目分别使用两种人脸库来进行比较分析。

\section{算法分析}

\section{程序实现}
\label{sec:shixian}

\section{结果分析}
\label{sec:jieguofenxi}




\end{document}







