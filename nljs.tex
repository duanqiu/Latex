\documentclass[a4paper,12pt]{ctexart}
\title{基于自发增强学习模型的脉冲神经网络以及在做决策中的应用}
\author{戴强}
\date{\today}
\usepackage{amsmath}
\usepackage{listings}
\usepackage{graphicx}
\usepackage{caption,subcaption}
\usepackage[top=1.25in,bottom=1in,left=1.25in,right=1.25in]{geometry}

\begin{document}
\makeatletter
\section{摘要}
受生物脉冲信息处理和多脑区协调机制的启发,本文提出了一种自治尖峰神经网络模型,用于决策。所提出的模型是具有自动环境感知的基底神经节回路的扩展。它自动从图像输入构造环境状态。 (1)在我们的模型中,简化的HodgkinHuxley计算模型是为了达到与LIF模型相近的计算效率而开发的,用来获得和测试认知中的离子水平性质。 (2)提出了一种基于尖峰的运动感知机制,在没有大量训练的情况下,从原始像素中提取学习过程的关键要素。我们将模型应用在“飞扬的小鸟”游戏中,经过数十次训练之后,它可以自动生成规则,在游戏中发挥出色。此外,我们的模型在霍奇金 - 赫胥黎(Hodgkin-Huxley)模型中阻断一些钠离子或钾离子通道的情况下模拟认知缺陷,这可以被认为是对深入离子水平的认知机制的计算探索

\section{介绍}

人类的大脑可以很好地处理复杂的任务,从视觉观察环境和情况到提供一个计划。 大脑可以自动分析和了解情况状况,提供适当的行为输出,而传统的人工智能系统缺乏自动的环境状态划分和理解。 受大脑结构和信息处理机制启发的算法可以为自动理解和解决问题提供更好的解决方案。

在本文中,我们构建了一个在做决策方面具有自主强化学习能力的脉冲神经网络。 我们通过测试它在计算机游戏“Flappy Bird”中获胜的能力来评估所提出的网络模型的优劣。 以原始图像作为输入,代理一开始就对环境知之甚少。 经过多次检测,它可以检测环境中的运动的物体,并避免与物体碰撞。 了解移动限制和环境,代理将学习如何移动以获得更好的游戏成绩。

我们的目的是建立一个类脑的强化学习的认知模型,并将其用于决策任务,使得具有这种模型的代理具有自主学习能力而不告诉环境是什么。 代理人可以通过提出的运动感知机制来感知环境,并用一些先验知识计算学习的关键要素。 我们希望我们的工作能够向类脑智能迈进一步。

\section{相关工作}
\subsection{基本神经中枢模型}
基础神经元模型是认知中最重要的脑结构之一,特别对于强化学习与决策。

基本神经元由The basal ganglia is composed of striatum, the STN, the GPe, and two output nuclei, the SNr and the GPi . Other nuclei, including the SNc and the VTA, are also considered as part of the basal ganglia. In cognition process, the basal ganglia need to cooperate with other related brain regions to form the  basal ganglia circuitry

从基本神经中枢的认知研究产生了大量的模型。 从基于大脑解剖学数据的基底节“盒子和箭头”模型开始[14,15],提出了许多计算模型[5,7-10,13,15,16]。 大部分的基本神经中枢模型是使用人工神经网络建立的[5,9,15],当在生物钟这些都是似是而非的。 另外,这些模型被用来做相对简单的动作选择实验,而不是自动地解决复杂的认知任务。

在本文中,我们将基于对基本神经中枢环路的理解,构建具有更多生物细节的自主学习模型。 在描述我们的模型之前,我们需要介绍基础神经中枢的数学模型,这些模型给了我们启发。


\subsection{基本神经元的Spike编码模型}

我们的工作与数学形式上的基础神经中枢动作选择模型和扩展的尖峰编码模型有关。
我们选择了基于脉冲的自主强化学习模型
(1)该模型具有较低的计算复杂度,保持了基本神经中枢的主要生物学特征。
(2)该模型可以处理数百个维度的广泛的输入。

脉冲编码模型的选择机制可以描述为下面的等式


其中f(xi)是原子核的输出,xi是0≤xi≤1之间的输入值。
基本神经中枢中的每个核由一组生物神经元代表。 一般认为,生物神经元通过产生复杂的脉冲序列携带信息。 这些峰值编码输入刺激作为燃烧率。 这个模型的编码和解码过程可以在[1,8,12]中找到。

\subsection{Hodgkin-Huxley模型}
Hodgkin-Huxley模型是1952年被Alan Lloyd Hodgkin和 Andrew Fielding Huxley提出的。他们根据对乌贼巨轴突的理解做出了这个模型。 使用实验数据拟合模型中的许多参数。 H-H模型被认为是描述动态电位在离子水平上动力学性质的最佳模型。 该模型被写成四个等式,I作为外部电流输入。
离子电流由三个组分组成。 即,具有三个活化门和一个钝化门的钠(Na +)电流,具有四个活化门的钾(K +)电流和主要由氯(Cl-)携带的泄漏电流。

在本文中,我们使用H-H模型来构建基本神经节模型,遵循上述的尖峰编码方法。 它可以模拟更多与认知有关的生物效应。 我们将把这个模型应用到真实的认知任务上来测试模型的离子性质,并模拟它对学习成绩的影响。

\section{类脑计算模型的自主强化学习}

\subsection{Hodgkin-Huxley模型的简化计算}
我们使用Hodgkin-Huxley模型去构造基于脉冲神经网络的增强学习,我们使用Hodgkin-Huxley模型有两个原因.(1)H-H模型准确地解释了实验结果,并能对神经细胞进行定量的电压分析。 (2)钠和钾与人类的认知有关。 钠或钾通道计算模拟离子电导的变化将使我们更深入地了解其认知的生物学性质。

基本神经中枢模型通常使用漏积分和火(LIF)神经元建立。 原因是LIF模型计算复杂度低,易于控制。 而LIF神经元对动态动作电位和离子水平活动的描述较差。 H-H模型深入到离子水平。

与LIF神经元不同,H-H模型更加精细。 如果LIF使用一步爬升到发射尖峰在1ms,由于离子通道的动态特性,H-H模型将在同一时间段内花费数百步到达阈值。 这个现象是由H-H方程确定的。 如果LIF模型以毫秒运行,则H-H模型以毫微秒运行。 这需要更多的模拟和更多的计算复杂性,因此,有必要为H-H模型提供简化的计算方程。

为了达到与LIF模型接近的计算效率,在本文中H-H方程进行了一些改变。 (1)膜电容C比原来的值小,设定为0.01〜0.03μF/ cm 2。这将大大增加一步行动的潜力。在达到阈值之前需要更少的步骤。 (2)为保证足够大的放电率,电流输入应大于6-7μA/ cm2,大电流输入可保证更高的准确度。根据我们的实验,对于当前输入来说,30的值是更好的选择。 (3)在H-H方程的模拟中没有动作电位积累的内部回路。前两个变化一步获得足够的电压增量。而且一步模拟得到的结果与以前相同。 (4)H-H方程在一定频率下自动产生尖峰脉冲。因此,为了控制H-H神经元的活性以编码不同的功能,执行干预机制。我们需要在尖峰后将电压重置为零,并根据所表示的函数和编码过程为方程分配一个不应期的时间段。


为了达到与LIF模型接近的计算效率,在本文中H-H方程进行了一些改变。 (1)膜电容C比原来的值小,设定为0.01〜0.03μF/cm2。这将大大增加一步行动的可能。在达到阈值之前需要更少的步骤。 (2)为保证足够大的放电率,电流输入应大于6-7μA/ cm2,大电流输入可保证更高的准确度。根据我们的实验,对于当前输入来说,30的值是更好的选择。 (3)在H-H方程的模拟中没有动作电位积累的内部回路。前两个变化一步获得足够的电压增量。而且一步模拟得到的结果与以前相同。 (4)H-H方程在一定频率下自动产生尖峰脉冲。因此,为了控制H-H神经元的活性以编码不同的功能,执行干预机制。我们需要在尖峰后将电压重置为零,并根据所表示的函数和编码过程为方程分配一个不应期的时间段。

为了达到与LIF模型接近的计算效率,在本文中H-H方程进行了一些改变。 (1)膜电容C比原来的值小,设定为0.01〜0.03μF/ cm 2。这将大大增加一步行动的潜力。在达到阈值之前需要更少的步骤。 (2)为保证足够大的放电率,电流输入应大于6-7μA/ cm2,大电流输入可保证更高的准确度。根据我们的实验,对于当前输入来说,30的值是更好的选择。 (3)在H-H方程的模拟中没有动作电位积累的内部回路。前两个变化一步获得足够的电压增量。而且一步模拟得到的结果与以前相同。 (4)H-H方程在一定频率下自动产生尖峰脉冲。因此,为了控制H-H神经元的活性以编码不同的功能,执行干预机制。我们需要在尖峰后将电压重置为零,并根据所表示的函数和编码过程为方程分配一个不应期的时间段。

我们将H-H模型的电压方程改写成如下,并对上述进行了一些改进:


其中J(x)是当前输入,原始输入x的线性函数,a和b是针对尖峰编码模型中不同神经元的随机值。 而30是当前输入的规模,以保证H-H模型的足够的发射速率。 峰值电压约为80-100mV。

通常,静止动作电位被设定为V = Vrest = 0以简化计算。 同时,其他电位和电导参数应给出以下值

当我们的模拟输入电流大于一个特定值,大约是6-7μA/cm2时,观察到有规律的尖峰活动[22]。 如果尖峰时间间隔为T,则平均发射速率可以表示为f = 1 / T,并且随着刺激增强[22-24]而增加,如图1所示。

图1霍奇金 - 赫胥黎模型在各种电流输入下的燃烧速率曲线在我们的实验中,当电流输入大于7.5μA/ cm2时,神经元开始具有稳定的正常尖峰。 静息电位设为0.时间常数为0.01 ms。 仿真周期为1秒。

\subsection{基于Spike的运动感知}
眼睛可以检测环境中的运动物体,检测结果将用于认知任务。 在本文中,我们建立一个神经网络模型作为代理人的眼睛。 具有这种模式的代理人可以在一定程度上理解和理解环境。 基础神经节的计算模型作为强化学习模型,代理人将获得自主学习能力,用图像输入做一些复杂的认知任务,而不是简单的动作选择。

图像的感知是在一个多层体系结构中进行的,如图2所示。它包含输入层,感知层和输出层。 对于图像输入,每个像素都有一组神经元来表示它在输入层的值。 使用种群编码算法将像素改变为一段时间的尖峰序列。 这些尖峰序列是感知层中两个特殊神经元的输入。 在这一层,两个神经元代表一个像素。 神经元A在当前工作,而神经元B在过去以恒定的时间延迟τ0工作。 这意味着神经元A接收当前的输入,而B接收最后的输入。 如果在两个神经元的位置有移动,他们将有不同的尖峰输出。 在输出层中,每个像素都有一个神经元。 神经元的输入是神经元A和神经元B.神经元A被激发,而B具有相反的效果。 如果没有移动,神经元没有脉冲,否则会释放脉冲。

整个运动感知过程如图3所示。图像中的每个像素都有其代表从输入编码层到输出层的神经元。 在输入编码层中,每个像素由一组神经元表示,并使用群体编码转换为脉冲序列[26]。 这些神经元具有重叠的高斯感知域,如图3所示。可以改变神经元的数量和编码间隔以获得良好的编码结果。 具有良好的参数,群体编码可以通过不同的尖峰模式来区分两个相似的值。 下一节的实验显示了它在像素编码方面的良好性能。

图3.原始图像输入的运动感知过程。 L0是原始图像,L1是输入编码层,L2是感知层,L3表示尖峰输出层,L4是原始图像的尖峰活动,表示运动区域。群体编码过程在L1中执行。 每个像素都被转换成一组神经元在L1中的尖峰序列。

在群体编码中,假设[Imin,Imax]是编码间隔,M是神经元的数目,每个神经元的高斯感知区域中心可以计算为

每个神经元的方差方程写为:

其中β的公共值在[0,2]处。 编码过程将计算一个值与每个高斯感知区域的交点,并获得这些高斯神经元的尖峰时间。

如图3中的L4所示,运动感知的输出是一个运动区域由尖峰突出显示的图像。在背景变化较小的环境中,很容易获得物体的运动方向,并将运动区域与其他背景区分开来。 这对于避免障碍物也是有用的,这将在下一节中进行描述。

\subsection{基于多区域协调和脉冲神经网络的自主强化学习模型}

在本节中,我们将描述我们的整个自主强化学习模型。 我们的自主模式旨在与环境进行交互,并进行强化学习,避免在不给环境状态的情况下移动障碍物。 本文所有的重要模型都是用Hodgkin-Huxley模型构建的。

我们的自主学习模型如图4所示。该模型形成一个Q学习循环,以学习与环境的交互。 运动感知模型用于从图像输入中检测运动物体并获得新的环境状态函数。 基础神经节模型将状态函数作为输入,并且在每个时间步骤中在所有可用动作之间实施行为选择。

图4.使用spiking神经网络建立的自主强化学习模型。 基础神经节结构及其动作选择功能(在矩形中)参考文献[1]和[2]。 f(d)是距离d的函数,称为状态函数。 Q值是状态函数和行为的函数。 环境状态由状态函数计算而不是手动创建。 除了几个基本的生存规则,代理人什么也不知道。

正如在Sect.2,使用[1,2]中的方法建立基本神经中枢模型。基底神经节回路的学习是通过更新基底神经节和前额叶皮层之间的突触连接来实现的。 学习不会完成,直到基底神经节获得良好的行动选择和代理人可以与环境良好的互动。

如果代理人要在环境中生存,就必须遵循两条规则:(1)代理人的正面方向不应有障碍。 (2)如果没有其他目标,代理人应尽可能远离障碍物。 状态函数和奖励按照这两个规则来计算。

\section{实验与应用}
在本节中,我们将我们的模型应用到称为“flappy bird”的游戏中,以测试我们的方法的性能。 下面我们首先对简化的H-H模型进行实验研究。 接下来,我们将在“flappy bird”游戏中评估我们的模型,

\subsection{简化版Hodgkin-Huxley模型}

我们提出了简化的H-H方程来降低计算复杂度,同时保持其离子性质。 图5中的实验显示了我们简化的H-H模型的性能。 根据实验结果,可以得出结论,简化的H-H模型具有较低的计算复杂度和与LIF模型相似的性能。

图5.简化H-H方程的实验。 答:原始H-H模型与其简化模型之间的仿真时间比较研究。 B:简化的H-H模型和LIF模型的函数跟踪实验。 结果表明,我们简化的H-H模型与LIF模型具有相似的函数恢复性能。

\subsection{在游戏中的自动强化学习}
在玩“flappy bird”这个游戏前,我们先来解释一下如何去计算环境状态函数和奖励。

在“飞扬的鸟”游戏中,管子向着鸟儿移动,鸟儿上下移动以调整位置,试图穿过管子的间隙而不会碰撞。 根据上面提出的两条存续规则,鸟应避免沿着其移动方向与管道碰撞,并尽可能远离每个管道的终端

环境状态是描述环境的一个关键要素。如果没有预先规定,它可以是与管道间隙中心的距离的函数,记为f(d)。 当一对管子到来时,鸟使用运动感知模型来获得管道间隙的中心。 环境状态函数f(d)是根据鸟的位置每一步计算的。 对于每个状态函数的值,鸟类可以采取两个上下的动作。 Q学习过程的Q值函数由f(d)和动作a表示,记为Q(f(d),a)。 每次更新Q值,错误都会被用于自治模型的学习。

奖励对学习很重要。 可以根据运动感知结果进行计算。 如果鸟的前方没有管道,奖励是积极的。 否则,奖励应该是消极的。

A.运动感知。 所提出的运动感知算法有助于检测障碍物的运动,根据尖峰结果计算出状态和奖励等关键信息。 运动感知结果如图6所示。

图6.演奏“flappy bird”的运动感知结果。 A:运动感知模型将当前时间步中的游戏场景作为输入,并在下一个时间步中给出尖峰结果。 B:每个人口编码步骤的尖峰结果。

尖峰结果显示了人口编码算法的良好输入编码性能。 编码过程在一定的时间间隔内将像素转换为尖峰序列。 这个时间间隔有几十个编码步骤。 从图6B可以看出,运动感知过程即使在一个单独的时间步中也能检测到主要的运动区域。 图6A中的尖峰结果是所有编码时间步骤的结果收集。

B.玩游戏。 我们在“flappy bird”游戏中训练自主学习模式。 实验结果如图7所示。从图7可以看出,一开始,鸟在碰撞管道前通常得分较低。 经过一番尝试之后,小鸟已经学会了如何避开管道,并在上次尝试中获得了30分以上的成绩,如果比赛还在继续,则分数会更高。 这个成就可以与一个好的人类玩家相比。


图7.使用我们的模型的游戏结果。 如果鸟与管道相撞,则需要重新尝试。 如果小鸟通过一个管道,将加1分。 A:游戏场景。 B:学习过程中的得分记录。 C:使用所提出的模型通过特定数量的管道的平均尝试次数


在图7C中,给出了使用所提出的模型通过特定数量的管道的平均尝试次数。 每个通过特定数量的管道的平均值是基于20次成功尝试。 值得注意的是,经过10次平均训练后,模型可以稳定通过10根管子,而对于20名人员,通过2根管子的平均训练次数为12.1次。另外,在训练过程中, 通过更多管道的70个管道似乎不需要更多的训练时间。

图8.在缺乏钠离子和钾离子的情况下学习的效果。 A:学习不含钠离子。 B:没有钾离子的学习。 C:不带钠离子弹“飞扬的鸟”的奖励和得分。 D:使用正常H-H模型的功能跟踪。


C.离子学习效果的计算验证。 钠离子和钾离子对生物大脑的认知过程非常重要1。 图8显示了在H-H模型中缺少这两个离子的情况下学习不足的一些模拟结果。 实验表明,没有钠离子或钾离子会导致学习认知障碍。 如图8C所示,即使训练时间较长,鸟只能通过一些管道,也不能获得较高的分数。 在这里,我们观察到钾离子的缺乏导致与在线自主强化学习期间钠离子的缺乏相比更差的情况。


\section{总结}

本文提出了一种基于峰值神经网络和多脑区协调机制的决策自治模型。 它可以做自主强化学习没有预定义的环境状态。

本文提出简化的霍奇金 - 赫胥黎(Hodgkin-Huxley)计算模型,使计算效率接近于LIF模型,同时保持离子水平性质的认知。 为了检测环境中的运动物体,开发了一种基于尖峰的运动感知机制,在没有大量训练的情况下从原始像素中提取关键要素。 在实验部分,我们将我们的模型应用于飞鸟游戏进行评估,表现出了良好的性能。 此外,我们的模型模拟霍奇金 - 赫胥黎(Hodgkin-Huxley)模型阻断钠离子或钾离子通道时的认知缺陷,这是深入到离子水平的认知的探索。

尽管有了这些贡献,但我们的模型仍然有一些局限性。 当背景随时间变化时,无法正确检测到移动物体。 模型本身仍处于高生物合理性的初始阶段。 在不久的将来,我们将引入更多有用的生物细节来改善我们的模型,以获得更好的认知表现。

\end{document}

